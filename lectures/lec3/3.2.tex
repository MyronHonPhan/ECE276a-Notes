\documentclass{article}
\usepackage{amsmath}
\begin{document}
\section{Euler Angle Parametrization}
You only need three angles (which we can achieve through rotations) to fully specify an object's orientation in 3D. We can denote these angles with respect to the 3
basis vectors in Euclidian space.\\

\noindent In general, we can apply these rotations either about the axes or on the axes. This is respectively
called \textbf{extrinsic axes} and \textbf{intrinsic axes}. Each of the extrinsic and intrinsic axes approaches can be further split
into the \textbf{Euler} (where you rotate one axis at a time) and \textbf{Tait-Bryan} (where you rotate all axes at once) Angles.\\

\noindent As a result, there are 24 ways that we can apply these rotations. This is because there are 3!=6 ways that you can apply the
sequence of rotations. From the multiplication rule in combinatorics, we multiply by 2 (for the two Euler and
Tait-Bryan angles) and multiply by 2 again (for the two extrinsic and intrinsic approaches) to get 24.
\begin{equation*}
	2*2*6=24
\end{equation*}
There is no practical difference to using extrinsic or intrinsic rotations. An extrinsic sequence of rotations is equivalent to an 
intrinsic set of rotations, just in reverse order.
\subsection*{Principal 3D Rotations}
An implicit representation (in essence, overparametrizing the rotation) of 3D rotations is using rotation matrices. if
we wanted to rotate about the X axis, then the rotation matrix of\\

\begin{equation*}
    X = 
    \begin{bmatrix}
    1 & 0 & 0 \\
    0 & \cos(\phi) & -\sin(\phi) \\
    0 & \sin(\phi) & \cos(\phi)
    \end{bmatrix}
\end{equation*}\\

\noindent Where we rotate an angle $\phi$ about or on the x-axis. We can perform similar rotations about/on the
y or z axis\\

\begin{equation*}
    Y = 
    \begin{bmatrix}
    \cos(\theta) & 0 & \sin(\theta) \\
    0 & 1 & 0 \\
    -\sin(\theta) & 0 & \cos(\theta)
    \end{bmatrix}
    \quad
    Z = 
    \begin{bmatrix}
        \cos(\psi) & -\sin(\psi) & 0\\
        \sin(\psi) & \cos(\psi) & 0 \\
        0 & 0 & 1
    \end{bmatrix}
\end{equation*}\\

\noindent These rotation angles ($\phi$,$\theta$,$\psi$), are called roll (X), pitch (Y) and yaw (Z). These terms and the order in 
which we specify them (X then Y then Z) is standard terminology in aerospace. We can combine these three rotations to 
achieve any orientation in 3D space.

\begin{equation*}
    R=    \begin{bmatrix}
        1 & 0 & 0 \\
        0 & \cos(\phi) & -\sin(\phi) \\
        0 & \sin(\phi) & \cos(\phi)
        \end{bmatrix}
        \begin{bmatrix}
            \cos(\theta) & 0 & \sin(\theta) \\
            0 & 1 & 0 \\
            -\sin(\theta) & 0 & \cos(\theta)
        \end{bmatrix}
        \begin{bmatrix}
            \cos(\psi) & -\sin(\psi) & 0\\
            \sin(\psi) & \cos(\psi) & 0 \\
            0 & 0 & 1
        \end{bmatrix}
\end{equation*}

\subsection*{Gimbal Lock}

Parametrization in terms of Euler angles can lead to interesting and annoying results. A result that we want
to avoid is called \textbf{gimbal lock}. This is a singularity that happens when we rotate by 90 degrees to a north
or south pole (on a metaphorical sphere). The result of this singularity is losing a degree of freedom.\\

\noindent Let's say we wanted to specify a sequence of intrinsic rotations starting with a 90 degree Y rotation. The
resulting rotation matrix would be:
\begin{equation*}
    Y = 
    \begin{bmatrix}
    0 & 0 & 1 \\
    0 & 1 & 0 \\
    -1 & 0 & 0
    \end{bmatrix}
\end{equation*}
If we applied that to the X rotation matrix, we would get this:
\begin{equation*}
    \begin{bmatrix}
        0 & 0 & 1 \\
        \sin(\phi) & \cos(\phi) & 0\\
        -\cos(\phi) & \sin(\phi) & 0 \\
    \end{bmatrix}
\end{equation*}
This actually corresponds with a rotation about the Z axis. We can easily show this through plugging in 90 degrees into
$\phi$ and applying it onto a unit vector in the x-direction:\\
\begin{equation*}
    \begin{bmatrix}
    0 & 0 & 1 \\
    1 & 0 & 0\\
    0 & 1 & 0 \\
    \end{bmatrix}
    \begin{bmatrix}
        1 \\ 0 \\ 0
    \end{bmatrix}
    =
    \begin{bmatrix}
        0 \\ 1 \\ 0
    \end{bmatrix}
\end{equation*}\\
This shows that we lose a degree of freedom if we pitch too much (by 90 degrees, to be exact) into the gimbal lock singularity.
We are quite literally unable to rotate in the x direction (only y and z are allowed) unless we pitched out of this singularity.
\end{document}